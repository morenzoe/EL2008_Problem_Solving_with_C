\documentclass[conference]{IEEEtran}
\usepackage{cite}
\usepackage{graphicx}

\title{Implementasi Algoritma Dijkstra}

\author{
\IEEEauthorblockN{Eraraya Morenzo Muten}
\IEEEauthorblockA{\textit{School of Electrical Engineering and Informatics}\\
\textit{Institut Teknologi Bandung}\\
Bandung, Indonesia\\
Emaill: 18320003@std.stei.itb.ac.id}
}

\graphicspath{{.report/graphics/}}

\begin{document}

\maketitle

\begin{abstract}
    Kebun Raya Purwodadi dengan luas area sekitar 85
hektar ternyata kekurangan papan informasi yang menyebabkan
pengunjung kerap kali kebingungan dalam mencari lokasi tanaman
tertentu. Paper ini bertujuan untuk membuat simulasi
dari algoritma yang dapat menentukan jarak terdekat antara
pengunjung (pengguna program) dengan lokasi tanaman yang
dituju. Algoritma yang digunakan adalah algoritma Dijkstra
yang beroperasi secara menyeluruh (greedy) untuk menguji
seitap persimpangan (Vertex) dan jalan (Edge) pada Kebun
Raya Purwodadi. Berdasarkan hasil simulasi dan pengujian,
kompleksitas ruang dari program ini adalah O(V) karena adanya
pembentukan array yang berisi V nodes untuk mencari heap minimum.
Sementara, kompleksitas waktu dari algoritma tersebut
adalah O(V2).
\end{abstract}

\begin{IEEEkeywords}
    component, formatting, style, styling, insert
\end{IEEEkeywords}

\section{Introduction}
Studi mengenai penggunaan algoritma Dijkstra dalam mencari
jarak terdekat dapat diimplementasikan pada kasus pencarian
tanaman pada Kebun Raya Purwodadi seperti yang telah
dilakukan oleh Yusuf et al di tahun 2017 [1]. Paper ini bertujuan
untuk melakukan simulasi kembali terhadap penelitian
yang telah dilakukan dengan bahasa C serta mengevaluasi
efisiensinya melalui perhitungan kompleksitas waktu dan ruang
dengan analisis Big-O.

\section{Arsitektur Sistem}
Di Kecamatan Purwodadi, Kabupaten Pasuruan, terdapat
salah satu kebun raya di Indonesia yang bernama Kebun
Raya Purwodadi yang memiliki luas area hingga 85 hektar.
Kebun raya sebagai fasilitas rekreasi dan penelitian ini ternyata
kekurangan papan informasi yang seharusnya disediakan oleh
pihak pengelola. Hal ini menyebabkan banyaknya pengunjung
yang merasa kebingungan untuk mencari lokasi dari tanaman
tertentu. Oleh karena itu, Yusuf et al (2017) memutuskan
untuk membuat suatu aplikasi dengan memanfaatkan algoritma
Dijkstra untuk membantu pengunjung Kebun Raya Purwodadi
dalam mencari lokasi tertentu.

%\begin{figure}[htbp]
%    \centering
%    \scalebox{0,6}{\input{graphics/Data Flow.pdf_tex}}
%    \caption{Arsitektur Sistem}
%\end{figure}

\section{Implementasi}
Di Kecamatan Purwodadi, Kabupaten Pasuruan, terdapat
salah satu kebun raya di Indonesia yang bernama Kebun
Raya Purwodadi yang memiliki luas area hingga 85 hektar.
Kebun raya sebagai fasilitas rekreasi dan penelitian ini ternyata
kekurangan papan informasi yang seharusnya disediakan oleh
pihak pengelola. Hal ini menyebabkan banyaknya pengunjung
yang merasa kebingungan untuk mencari lokasi dari tanaman
tertentu. Oleh karena itu, Yusuf et al (2017) memutuskan
untuk membuat suatu aplikasi dengan memanfaatkan algoritma
Dijkstra untuk membantu pengunjung Kebun Raya Purwodadi
dalam mencari lokasi tertentu.

\subsection{Implementasi Graph pada Array dalam Bahasa C}
The IEEEtran class file is used to format your paper and style the t

\section{Kesimpulan}
This document is a model and instructions

\bibliographystyle{IEEEtran}
\bibliography{references.bib}

\end{document}